\documentclass[12pt]{article}

\usepackage{graphicx}
\usepackage{amsmath}
\usepackage{geometry}
\geometry{margin=1in}

\title{Autocorrelation in Florida Weather}
\author{Group 3 Cool Coatis}
\date{Dec 2025}

\begin{document}
\maketitle

\section*{Introduction}

In this project, we looked at whether Florida’s annual mean temperatures show any year-to-year dependence. This allows us to discover whether warm years tend to be followed by warm years more often than expected by chance. 

\section*{Methods}

We analysed the dataset \texttt{florida\_weather.csv}. This dataset contains annual mean temperatures from 1901 to 2000 in Florida. 

To quantify temporal dependence, we used the Pearson correlation to calculate
between temperature in year $t$ and year $t+1$:
\[
r_{\text{obs}} = \operatorname{cor}(\mathrm{Temp}_t, \mathrm{Temp}_{t+1}).
\]

Because temperature data over time may not satisfy the assumptions of the usual correlation test, we used a permutation test instead. We randomly permuted the temperature
values 5000 times. This recalculated the successive-year autocorrelation for each
permuted sequence to construct a null distribution. 


\section*{Results}

The observed autocorrelation was
\[
r_{\text{obs}} = 0.3262.
\]

The null distribution of correlations obtained from 5000 permutations
was near zero. Only 0.0002 of the permuted correlation values were
greater than or equal to the observed value, resulting in a one-sided
p-value of 0.0002.
\[
p = 0.0002.
\]

This provides strong evidence for positive temporal autocorrelation in
Florida’s annual temperatures. This successfully supports that warm years tend to be followed by warm years more often than expected under a random ordering of the same values.

\begin{figure}[h!]
    \centering
    \includegraphics[width=0.75\textwidth]{../results/autocorr_hist.pdf}
    \caption{
    Null distribution of lag-1 autocorrelation coefficients generated
    from 5000 random permutations of annual mean temperatures. 
    }
\end{figure}

\section*{Conclusion}

This analysis shows significant successive-year autocorrelation in the Florida
temperature time series. The vertical line shows the observed correlation r_obs = 0.3262.

\end{document}